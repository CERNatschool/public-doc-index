%%%%%%%%%%%%%%%%%%%%%%%%%%%%%%%%%%%%%%%%%%%%%%%%%%%%%%%%%%%%%%%%%%%%%%%%%%%%%%
\section{Introduction}
\label{sec:intro}
%%%%%%%%%%%%%%%%%%%%%%%%%%%%%%%%%%%%%%%%%%%%%%%%%%%%%%%%%%%%%%%%%%%%%%%%%%%%%%
CERN@school is a student-led research programme that aims to bring
\acs{CERN} into the classroom.
It is largely based around the
Timepix hybrid silicon pixel
detector~\cite{Timepix2007,TimepixErratum2008},
a device developed by the Medipix2 Collaboration\footnote{%
See \url{http://medipix.web.cern.ch}}
that enables the user to visualise and record measurements of
ionising radiation in real time.
As well as changing the way radiation can be explored in the classroom,
data from Timepix devices deployed around the UK, in space, and
at the \ac{LHC} can also be used by
CERN@school collaboration members to carry out authentic, novel scientific
research suitable for publication in peer-reviewed journals.
%

A brief history and description of the programme may be found in
the proceedings of the 37$\thsuper$ International Conference on
High Energy Physics (ICHEP 2014)~\cite{Whyntie2016b}.
%
Since that conference, held in July 2014, CERN@school has become a 
flagship programme of the
Institute for Research in Schools\footnote{%
See \url{http://researchinschools.org}} (IRIS),
a UK charity that aims to offer schools students and their teachers
the opportunity to participate in real science in many different fields.
%
This document aims to capture the progress and status of the
CERN@school programme in the period from June 2012
to December 2016.
This roughly corresponds to the start of the
\ac{STFC}
Science in Society Large Award (grant number ST/J000256/1)
and the end of the
\ac{STFC} \ac{PEF} for CERN@school
(grant number ST/N00101X/1).
%
To this end, the documents, presentations,
events, code repositories, and publications associated with
CERN@school research, educational, and promotional activities
have been indexed and are briefly described in the relevant
sections and subsections of this document.
%
Hyperlinks and \acp{DOI} are provided where
available.
%

%=============================================================================
\subsection{Overview of the index}
\label{sec:introoverview}
%=============================================================================
Section~\ref{sec:cas} indexes the
documents (Section~\ref{sec:casdocuments}),
presentations (Section~\ref{sec:caspresentations}),
events (Section~\ref{sec:casevents}),
and
code repositories (Section~\ref{sec:cascode})
related to the core CERN@school programme.
This includes research-based and educational material related to
the Timepix detector that has been produced as a direct result of the
programme, such as the
\acf{LUCID} %Langton Ultimate Cosmic ray Intensity Detector (LUCID)
satellite experiment,
\acf{RAY}, %Radiation Around You (RAY),
and
the \acl{UKSA}-backed TimPix Principia project.
%

Section~\ref{sec:moedal} provides the same information
(Sections~\ref{sec:moedaldocuments}--\ref{sec:moedalcode})
for items relating to the \ac{MoEDAL} experiment.
This material has been given its own section as
%CERN@school/IRIS joined MoEDAL, and
generally speaking the work reported linked to the \acs{STFC} \acs{PEF}
and was therefore more research-focussed.
%
The peer-reviewed publications produced as part of
CERN@school and/or \acs{MoEDAL} have been listed separately
in Section~\ref{sec:publications}.
%
References, acronyms, and acknowledgements may be found in
sections~\ref{sec:references},~\ref{sec:acronyms},
and~\ref{sec:ack} respectively.
