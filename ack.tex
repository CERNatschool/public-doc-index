%%%%%%%%%%%%%%%%%%%%%%%%%%%%%%%%%%%%%%%%%%%%%%%%%%%%%%%%%%%%%%%%%%%%%%%%%%%%%%%
\section{Acknowledgements}
\label{sec:ack}
%%%%%%%%%%%%%%%%%%%%%%%%%%%%%%%%%%%%%%%%%%%%%%%%%%%%%%%%%%%%%%%%%%%%%%%%%%%%%%%
CERN@school has grown from a single school's visit to \acs{CERN}
into a nationwide flagship programme for the \acf{IRIS},
via a regional pilot scheme and national expansion supported by
\ac{SEPnet} and \ac{STFC} respectively.
This would not have been possible without the support of a great number of
people and organisations who have seen the potential of the programme
and what its philosophy offers participating students and teachers.
While the following list is not exhaustive, it is hopefully indicative of
the tremendous amount of help that has been offered and provided
by scientists, experts, and educational practioners from around the world.

Firstly, there would be no CERN@school without the
Medipix Collaboration. The Timepix detector is a phenomenal piece of
research-grade equipment that has made, and will continue to make,
a huge impact in many fields
that require real-time visualisation of ionising radiation.
Science education is undoubtedly one of these fields.
M. Campbell, spokesperson for the Medipix Collaboration,
has provided a huge amount of support from day one
and has been, quite literally, instrumental in its success.
We also thank J. A. Alozy for support with detector repair
and maintanence, J. Id\'{a}ragga, M. Benoit and S. Arfaoui
for the GEANT4-based AllPix simulation suite used
in~\cite{Whyntie2014,Whyntie2015a,Whyntie2015b},
and the whole of the Medipix Collaboration for their
hospitality and engagement at various collaboration meetings.
%
CERN, likewise, has been very supportive of the programme.
We are grateful for the continued permission to use the
\acs{CERN} name and logo for CERN@school activities and research,
and would like to thank J. Gillies, R. Vanden Broeck, and K. Kahle
for their help engaging with CERN and beyond.
The author would also like to thank the Users Office,
IT Department, Hostel service, and the staff in Restaurant 1
for providing a consistently excellent working environment\footnote{%
And exceptional croissants fourr\'{e} cr\`{e}me p\^{a}tissi\`{e}re.}.
There truly is nowhere like it in the world.
%

The support and guidance of the UK \acf{STFC} has been essential to
the sustained growth of CERN@school, both in terms of the team and
equipment. The \acs{STFC} Public Engagement group's
Science in Society programme and Public Engament Strategy
have supported CERN@school through the very successful Large Award and
\acf{PEF} schemes, as well as additional
Capital Resource allocations.
Furthermore, the guidance and oversight provided by the
\acs{STFC} Steering Group for CERN@school ensured that the programme
ran as it should:
we would like to thank
R.~Clegg,
E.~Cunningham,
D.~A.~Gillespie,
N.~Hollingworth,
A.~Thompson,
and
C.~Woolford
from the \acs{STFC} Public Engagement team,
and
J.~P.~Eastwood,
C.~Johnson, % CHECK
C.~P.~R.~Thorley,
and the various educators who joined the Steering Group for this.
A special note of thanks should also go to
J.~Womersley, % CHECK 
former Chief Executive of \acs{STFC}, % CHECK
for supporting CERN@school so enthusiastically during his time in the role.
%
Likewise, the contribution of the
Royal Commission for the Exhibition of 1851 via a Special Award
ensured funding for the Fellow continued after the Large Award ended and
the \ac{PEF} began; thanks are due to N.~Williams and
his team with respect to this.

The Langton Star Centre,
and the Simon Langton Grammar School for Boys\footnote{%
Girls are admitted in the sixth form.},
provided an ideal environment for CERN@school to grow and flourish.
%
The support, freedom, and trust students are given to pursue their own
lines of investigation their have been paramount to the success of the
student-led projects that have started there.
However, this would not be possible without a fantastic team of
staff working tirelessly behind the scenes to make sure the
research and educational needs of the students are met.
The author is grateful to the following for their help:
R.~Champion, T.~Connolly (educational support);
S.~Begg, J.~Bradley, S.~Greenwood,
N.~Hatton, H.~Mudhar, E.~Mudhar-Fung,
J.~Poyner, and D.~Wash (adminstrative support);
I.~Begg, C.~Boucher, T.~Forester, and M.~Kemp (technical support);
and
M.~N.~F.~Baxter and K.~Moffat (Senior Management Team).
%
A special note of thanks must go to G. Poole who led the
Science Department during the author's time at the school,
and provided invaluable guidance and the secure educational
foundations upon which the Langton Star Centre (and so CERN@school)
could be built.

The CERN@school network could not have expanded as it did without
two key organisations: the \acf{SEPnet} and the UK \acf{IOP}.
The former successfully piloted a regional programme via
their university-based outreach officers, while the latter
expanded the network across the UK and Ireland through their
\acf{PTN}.
We thank
C.~Harvey
and
J.~West
of \acs{SEPnet}
and
C.~Shepherd
and
G.~Williams
of the \acs{IOP}
for their continued support and expertise.
V.~Benson (\acs{SEPnet})
and
V.~Fox (\acs{IOP})
also provided a number of exceptional summer students via their
respective internship programmes.
The contributions of A.~Coupe, J.~Cook, and N.~Shearer
have been noted elsewhere in this document;
they were a pleasure to supervise and the author wishes them
all the best in their future careers.
%
There are too many outreach officers and \acfp{PNC}
to mention by name, but likewise their contributions have been
greatly appreciated.
%
The author also wishes to acknowledge the support of
A.~Newsam (Liverpool John Moores University)
provided through the \acs{IOP} mentorship scheme.

Many of CERN@school's sub-projects have benefitted
from expert advice and guidance. These include:
\bullettext{LUCID} -- D.~Cooke, E.~Brownbill, S.~Ghanoun,
V.~O'Donovan, and S.~Wokes (\acs{SSTL}),
D.~J.~Garton (Satellite Applications Catapult),
J.~Jak\r{u}bek (\acs{CTU} Prague, detector calibration),
and J.~P.~Eastwood (Imperial College London);
the \bullettext{detector network} -- V.~Stanislav (Jablotron),
J.~R.~Wilson, C.~P.~R.~Thorley, and B.~Frost (\acs{QMUL}),
H.~Pollard (The Ogden Trust),
I.~Holmes (Kettering Buccleuch Academy),
S.~Palmer, L.~Mowberry, V.~Stowell, and J.~Lewis (\acs{RAL} Public Engagement),
L.~F.~Thomas (LFT Consulting) and G.~Steele (\acs{SSERC});
\bullettext{TimPix} -- L.~S.~Pinsky (\acs{NASA}/Uni. of Houston),
N.~N.~Stoffle (\acs{NASA}), Son Hoang (Uni. Houston),
L.~F.~Thomas (LFT Consulting -- again!),
L.~Jackson, J.~Curtis, and H.~Garrett (\acs{UKSA}),
and T.~Peake (\acs{ESA}).

Involvement with the \acs{MoEDAL} experiment started as a sub-project,
but as the \acs{STFC} \acs{PEF} (and CERN@school as a whole) evolved
so did the opportunities for extending the scope of the research activities.
This is largely down to the support provided by the \acs{MoEDAL}
Collaboration as led by J.~Pinfold.
Many people have provided help in the following areas:
R.~Soluk (Uni. Alberta) -- Timepix detector installation;
D.~Felea (ISS, Bucharest) -- simulations and detector geometry;
S.~Posp\'{i}\v{s}il, B.~Bergmann, and P.~Bene\v{s} (\acs{CTU} Prague) -- the
\acs{CTU} Timepix detector network;
and
V.~Wedlake (\acs{CERN}) -- adminisitrative support.
Special thanks should also go to
P.~Mermod (Uni. Geneva) for successfully chairing the
\acl{SaAG} and leading the efforts that
led to \acs{MoEDAL}'s first physics publications from the
\acs{MMT} subdetector results~\cite{MoEDAL2016a,MoEDAL2017a},
and to the various CERN Summer students who contributed
to \acs{IRIS} \acs{MoEDAL} projects.

On the computing front, the author would like to thank
the CernVM group (\acs{CERN}) for developing the CernVM
and CernVM-FS services that have made access to Grid resources
much, much easier for schools and other new user communities.
%
From GridPP, the following people have provided invaluable support
for CERN@school:
C.~Condurache (RAL) -- CernVM-FS;
%has supported GridPP's CernVM-FS
%capabilities at the Rutherford Appleton Laboratory (RAL);
%
D.~Bauer, S.~Fayer (Imperial College London),
and J.~Martyniak (now at RAL) -- GridPP DIRAC;
%developed and supporting GridPP DIRAC;
%
the Ganga development team\footnote{%
See \href{https://github.com/ganga-devs/}{https://github.com/ganga-devs/}};
%for their work on and support for the Ganga user interface;
S.~Skipsey (Uni. Glasgow) and J.~Jensen (RAL) -- data storage;
%from the
%GridPP Storage group for VO storage-related support;
%
A.~McNab (Uni. Manchester) -- web support;
N.~O'Neill (previously at QMUL) -- outreach;
%with technical support with
%the GridPP website and online user guide\footnote{%
%See \href{https://www.gridpp.ac.uk/userguide/}{https://www.gridpp.ac.uk/userguide/}};
%
D.~Taylor, A.~Owen, C.~Timis, T.~Froy, and C.~J.~Walker (QMUL) -- QMUL Tier-2 cluster support;
%support with development work on the Queen Mary University of London cluster;
%
J.~Coles (Uni. Cambridge) -- new user community support;
and
%for leading GridPP Operations and
%coordinating new user community activities;
%
P.~Gronbech (Uni. Oxford) -- project management.
%for support with Project Management;
%
%
Special thanks also go to
D.~Britton (Uni. Glasgow) for leading the GridPP Collaboration
and supporting the vision of integrating CERN@school via
the New User Engagement Programme
as it developed,
and
%
S.~L.~Lloyd (QMUL) for chairing the GridPP Collaboration Board,
providing numerous testing and support mechanisms for Grid
activities, initiating CERN@school's involvement with GridPP,
and providing much support and guidance as the two
organisations worked together.

Last, and by no means least, the author would like to thank
B.~Parker for leading CERN@school, the Langton Star Centre,
\acs{IRIS}, and the many other projects that have inspired
huge numbers of students in many areas of science and beyond.
It has been an incredible experience to work with someone the
author will always describe as the
``fifth fundamental force of Nature'',
and the impact she has made on the countless students,
educators, and indeed scientists through her work is
%(and perhaps always will be)
immeasurable.

{\em The work described here was supported by the
UK \acf{STFC} via grant numbers ST/J000256/1 and ST/N00101X/1,
and a Special Award from the Royal Commission for the Exhibition of 1851.
%
The CERN@school Collaboration would also like to acknowledge the
support provided by
the GridPP Collaboration~\cite{GridPP2006,GridPP2009}
in terms of both computing resources and technical guidance
from collaboration members.}
