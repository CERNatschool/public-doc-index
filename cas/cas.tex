%%%%%%%%%%%%%%%%%%%%%%%%%%%%%%%%%%%%%%%%%%%%%%%%%%%%%%%%%%%%%%%%%%%%%%%%%%%%%%
\section{The CERN@school programme}
\label{sec:cas}
%%%%%%%%%%%%%%%%%%%%%%%%%%%%%%%%%%%%%%%%%%%%%%%%%%%%%%%%%%%%%%%%%%%%%%%%%%%%%%
As noted in the introduction, an overview of the CERN@school
programme may be found in the proceedings of the 2014
\ac{ICHEP}~\cite{Whyntie2016b}. A brief summary of the programme's
activities is provided below; the reader may then find further information
in the listings of
documents, %(Section~\ref{sec:casdocuments}),
presentations, %(Section~\ref{sec:caspresentations}),
events, %(Section~\ref{sec:casevents}),
and
code %(Section~\ref{sec:cascode})
(subsections~\ref{sec:casdocuments},
\ref{sec:caspresentations},
\ref{sec:casevents},
and~\ref{sec:cascode} respectively).
%
\begin{itemize}
\item \bullettext{In the beginning}:
CERN@school was conceived when students from the
Simon Langton Grammar School for Boys
visited the Medipix laboratories at \acs{CERN} in 2007.
%
After seeing the potential for using the
Timepix hybrid silicon pixel detectors~\cite{Timepix2007,TimepixErratum2008}
for detecting and visualising ionising radiation in the
classroom,
a pilot project was established with the support of
the \acf{SEPnet}
and
Kent County Council (KCC)
that placed ten detectors in schools around Kent, England.
%
\item \bullettext{LUCID}: 
Students from the Langton Star Centre,
the research facility attached to the school and
directed by B. Parker,
then submitted a proposal for the
\acf{LUCID}
experiment to a satellite competition for schools
run by the British National Space Centre\footnote{
  Now the \acf{UKSA}.
} and
\acf{SSTL}.
%
Consisting of five Timepix detectors arranged in an
open-faced cube, \ac{LUCID} was ultimately accepted as one of
scientific payloads aboard SSTL's \ac{LEO} technology demonstration
satellite, TechDemoSat-1.
%
In 2011, the UK \acf{STFC}
awarded \ac{LUCID} a Science in Society Large Award to support the
dissemination of data and the development of educational and
promotional resources for schools.
%
In the run-up to the (delayed) launch,
GEANT4~\cite{GEANT42003,GEANT42006}
simulations of LUCID were performed to estimate the performance
of the detector and provide training data for CERN@school students.
The results of this work were presented at the
\acs{CHEP} 2013~\cite{Whyntie2014} and \acs{iWoRID} 2014~\cite{Whyntie2015a}
conferences.
%
On Tuesday the 8th July 2014,
TechDemoSat-1 launched successfully aboard a Soyuz 2b
launch vehicle from Baikonur Cosmodrome, Kazakhstan. The satellite has
since been commissioned and has been delivering data for the CERN@school
programme since October 2014. LUCID project leader
C. Hewitt presented the first results at the 2015 \acf{NAM}
(see section~\ref{meeting:nam2015lucid}).
%
\item \bullettext{The Detector Network}:
With the launch of TechDemoSat-1 delayed
and additional funding secured,
the opportunity was taken 
to expand the original CERN@school detector network.
Twenty-five Timepix detectors,
supplied by Jablotron\footnote{%
  See \texttt{http://www.jablotron.com}},
were procured and distributed around the UK
via the \acf{IOP}
\acf{PTN} (see section~\ref{meeting:iop2013maydetdist}).
%
Many presentations and workshops were given to introduce
students, teachers, and outreach network coordinators
to the Timepix technology and data management systems.
%
In 2013, a Royal Commission for the Exhibition of 1851
Special Award was made to support the role of
Schools Research Champion for this national network of
detectors (and \ac{LUCID}).
%
Subsection~\ref{meeting:eclipse2015event} describes the
CERN@school Eclipse 2015 event, where many of the Timepix
detectors in schools made measurements before, during, and
after the total solar eclipse of Friday the 20th of March
2015~\cite{Whyntie2016a}.
%
Results from this experiment, and many others including
the \ac{RISE}, \ac{RAY}, and \ac{FAIR}, were presented at the
three CERN@school Research Symposia held at
the University of Surrey (2014, section~\ref{meeting:cassymposium2014}),
\acl{QMUL} (2015, section~\ref{meeting:cassymposium2015}),
and
the \acl{RAL} (2016, section~\ref{meeting:cassymposium2016}).
%
\item \bullettext{The demonstration experiments and user guides}:
as the Timepix detector is a research-grade piece of scientific
equipment, four classroom demonstration experiments were carried out,
written up and published in peer-reviewed journals -- {\em Physics
Education}~\cite{Whyntie2013} and
{\em Contemporary Physics}~\cite{Whyntie2015b}.
This was to demonstrate not only the physics but how to prepare
and publish one's scientific work and results with CERN@school.
User guides and instructions were also provided on the 
CERN@school website, which was at the time hosted by \acs{CERN}\footnote{%
Formerly at \href{http://cernatschool.web.cern.ch}{http://cernatschool.web.cern.ch}}.
The datasets from the experiments have also been made available
in the \href{http://fighare.com}{FigShare} online data repository
(see Section~\ref{sec:casdocuments}).
Following feedback from teachers and outreach officers,
a new set of guides and demonstration experiment worksheets were
prepared for \acs{IRIS} by LFT Consulting and the \acf{SSERC}.
CERN@school also moved its web presence to the \acs{IRIS} website,
where these documents are hosted (see Section~\ref{sec:casdocuments}).
%
\item \bullettext{Coding with CERN@school}:
The relatively simple data format used by the Timepix detector -- \code{ASCII}
text files containing the pixel $x$, $y$, and count values -- meant
that data processing and analysis with high-level programming languages like
Python could be done fairly easily.
CERN@school could therefore provide students (and teachers) with an
introduction to coding motivated by scientific research
(as opposed to making an app or other more trivial applications).
To this end, various training materials featuring CERN@school
data were prepared (see Section~\ref{sec:casdocuments}).
A CERN@school \href{http://github.com}{GitHub} educational
organisation was also created and CERN@school-related code was published there,
including the code used in the published CERN@school experiments.
See Section~\ref{sec:cascode} for further information and a list of
relevant code repositories.
%
\item \bullettext{GridPP and the CernVM}:
in October 2013, the Fellow joined the
GridPP Collaboration~\cite{GridPP2006,GridPP2009} as Dissemination
Officer.
%
GridPP, which represents the UK's contribution to the
\ac{WLCG}, had previously offered computing support for
CERN@school via the \texttt{cernatschool.org} \ac{VO}.
This appointment offered the opportunity to formalise this
partnership. \texttt{cernatschool.org} acted as a
technology demonstrator \ac{VO} for GridPP's
New User Engagement Programme.
By using these technologies
to power the work featured in CERN@school publications,
e.g.~\cite{Whyntie2015a}, they were tested and ready for
other new user communities
from medical physics, astrophysics and computational biology
to use in their activities and increasing GridPP's engagement
beyond the \ac{LHC} and particle physics.
Further details and case studies (of which CERN@school is one)
can be found in the document described in
Section~\ref{meeting:gridppcasestudies}.
CERN@school students were also involved in testing and providing
feedback on the GridPP UserGuide, the document described
in Section~\ref{meeting:gridppuserguide} which provides a
complete guide to getting on the Grid with GridPP.
Furthermore, the GridPP CernVM -- a \ac{VM} image provided
by \acs{CERN} configured for use with the GridPP New User
Engagement Programme -- can provide an out-of-the-box
working environment for coding with Python and
CERN@school. A document describing how to create a GridPP CernVM
for use with CERN@school is described in Section~\ref{meeting:cascernvmcheatsheet}.
CERN@school was also represented at many of the GridPP Collaboration
meetings in this period; see Section~\ref{sec:caspresentations}
for further details.

%
\item \bullettext{TimPix and Principia}:
following a successful partnership for a \acs{UKSA}-funded
\acs{LUCID} side project for primary schools,
``{\em Whatever the Space Weather}'',
L. F. Thomas (LFT Consulting) worked on behalf of
the Langton Star Centre and then the \ac{IRIS} to establish the
TimPix project.
Part of the \ac{UKSA} Principia Mission,
which coordinated the outreach activities associated with
UK \ac{ESA} astronaut Tim Peake's visit to the \ac{ISS},
TimPix gave participating students access to data from the
five Timepix detectors installed aboard the \ac{ISS}.
Students working on TimPix projects presented their work
at the Summer 2016 Medipix Collaboration meeting (Section~\ref{meeting:mpxcol2016cern})
and the Principia Schools Conference (Section~\ref{meeting:principia2016pompey}).
A CERN@school Timepix detector also featured in a demonstration
of real-time radiation visualisation in Dr Kevin Fong's
2015 Royal Institution Christmas Lectures (Section~\ref{meeting:rixmaslectures2015}).
%
\item \bullettext{The MoEDAL experiment at CERN}:
owing to the students' experience with the Timepix detector,
the Langton Star Centre joined the \ac{MoEDAL}
collaboration in 2013\footnote{%
See \href{https://cds.cern.ch/journal/CERNBulletin/2013/48/News Articles/1630194}{CERN Bulletin, Issue No. 48--49/2013}.},
and so CERN@school added data from a \acf{LHC} experiment
to its scientific portfolio\footnote{%
CERN@school@CERN, if you will.}.
However, as this became the focus of the Fellow's activities from
January 2015, the documents, presentations, events and code
repositories have been given a dedicated section
(Section~\ref{sec:moedal}).
\end{itemize}


%\clearpage


%=============================================================================
\subsection{Documents}
\label{sec:casdocuments}
%=============================================================================
Table~\ref{tab:casdocuments} lists the documents relating to
CERN@school research and activities,
which are then briefly described in the following subsections.
%
This includes things like internal reports, posters, experimental protocols,
published datasets, and user guides, but not
presentation slides (which are included with the
relevant presentation in Section~\ref{sec:caspresentations})
or code repositories (which are described in Section~\ref{sec:cascode}).
%
Peer-reviewed publications from the CERN@school programme have been
listed in Section \ref{sec:publications} and are marked with an
asterisk in Table~\ref{tab:papers1}.

\begin{landscape}

\input{cas/documents/documentstable}

\end{landscape}

\input{cas/documents/documents}

\clearpage

%=============================================================================
\subsection{Presentations}
\label{sec:caspresentations}
%=============================================================================
Table~\ref{tab:caspresentations} provides a list of selected presentations
relating to the CERN@school programme from June 2012 to December 2016,
i.e. the period covering the STFC Large Award, the Special Award from
the Royal Commission for the Exhibition of 1851, and the
CERN@school STFC Public Engagement Fellowship.
These are then detailed in the following subsections,
except for standard school visits (i.e. talks in schools where an overview of
the programme and basic detector and data training was provided).
Note that MoEDAL-related presentations are featured in
Table~\ref{tab:moedalmeetings} (Section~\ref{sec:moedalpres}).

\input{cas/presentations/presentationstable}

\clearpage

\input{cas/presentations/presentations}

\clearpage

%=============================================================================
\subsection{Events}
\label{sec:casevents}
%=============================================================================
Table~\ref{tab:casevents} lists events that took place between
June 2012 and January 2017 that in some way involved CERN@school.
Each event is then briefly described in the following subsections.
Please note that,
generally speaking,
if there was a presentation given at an event
it will be listed in Section~\ref{sec:caspresentations}
and may not be listed here.

\input{cas/events/eventstable}

\clearpage

\input{cas/events/events}

\clearpage

%=============================================================================
\subsection{Code repositories}
\label{sec:cascode}
%=============================================================================
In order to ensure students could freely participate in
CERN@school-related scientific studies,
all software and analysis code required to analyse data from the Timepix
detectors was made publicly available via the online code repository
\href{http://github.com}{GitHub}.
When CERN@school existed as a separate entity,
CERN@school used the CERN@school GitHub Organization:

\href{http://github.com/CERNatschool}{http://github.com/CERNatschool}

as a place for hosting software repositories that were undergoing active development.
GitHub kindly provided an Educational Account for CERN@school
which meant that a limited number of private repositories could be kept too.
%
However, with the creation of the \acf{IRIS}, a new GitHub Organization
was created for software development across all \ac{IRIS} projects:

\href{http://github.com/InstituteForResearchInSchools}{href://github.com/InstituteForResearchInSchools}

and the CERN@school Organization has since been decommissioned.
As a number of the repositories featured code used
in the CERN@school documents and publications,
the repository will remain on the GitHub website but will no longer
be active.
Except where otherwise noted,
the software is issued under an MIT software license
and may be used by \acs{IRIS} students (or indeed anyone else)
as required.
Furthermore, stable releases of these repositories
have been published on the \acs{CERN}-backed
\href{http://zenodo.org}{Zenodo} repository
which assigns them a permanent \ac{DOI}.
The relevant repositories are listed in
Table~\ref{tab:cascode} and detailed in the
following subsections.

\input{cas/code/codetable}

\clearpage

\input{cas/code/code}

\clearpage
