%%%%%%%%%%%%%%%%%%%%%%%%%%%%%%%%%%%%%%%%%%%%%%%%%%%%%%%%%%%%%%%%%%%%%%%%%%%%%%
\section{MoEDAL}
\label{sec:moedal}
%%%%%%%%%%%%%%%%%%%%%%%%%%%%%%%%%%%%%%%%%%%%%%%%%%%%%%%%%%%%%%%%%%%%%%%%%%%%%%
The \acl{MoEDAL}~\cite{MoEDAL2009}\
is the seventh major experiment at \acs{CERN}'s \acf{LHC}.
It is housed underground at Interaction Point (IP8), sharing
the experimental cavern with the \acs{LHCb} experiment.
%
\acs{MoEDAL} is designed to probe the \acs{LHC}'s particle collisions
for signs of Paul Dirac's hypothesised magnetic monopole,
as well as other highly ionising signs of new physics that
the other \acs{LHC} experiments cannot easily look for.
%
A full introduction to the \acs{MoEDAL} experiment may be found
in document CAS-PUB-MDL-000002 (see Section~\ref{meeting:moedalcasguide})
and an accompanying bibliography in CAS-PUB-MDL-000001
(see Section~\ref{meeting:moedalbib}); the reader is advised to
consult these for more information about \ac{MoEDAL} itself.
%

The \acf{IRIS} is a full member of 
the \acs{MoEDAL} Collaboration, with the Langton Star Centre having
joined in 2013 owing to their experience with the Timepix detectors.
Students and teachers associated with \acs{IRIS}
have access data from the \acs{MoEDAL} experiment and are able to
contribute to the \acs{MoEDAL} Collaboration's research programme.
%
With the expansion of the CERN@school team under \acs{IRIS},
it was decided that from January 2016
the Fellow should focus on research activities
with the \acs{MoEDAL} Collaboration. These included:

\begin{itemize}
\item \bullettext{Collaboration activities}:
as with any \acs{LHC} experiment, there were many collaboration activities
that the Fellow was required to perform as part of the role.
These included the biannual Collaboration Meetings
and various trips to \acs{CERN}.
As a member of the \ac{SaAG},
there were also the \ac{SaAG} meetings every other week.
All of these meetings, with brief summaries,
are listed in the Presentations and/or Events sections.
Furthermore, there were various promotional activities
undertaken on behalf of the Collaboration. These included the
Monopole Quest! exhibit at the Royal Society Summer Science Exhibition
(see Section~\ref{meeting:rssse2015}),
various schools talks,
and an article published via The Conversation which was read by
over half a million people (see Section~\ref{meeting:moedalconv}).
%
\item \bullettext{The Timepix detector network}:
funding for two Timepix detectors for deployment in the \acs{LHCb}
cavern was secured from \acs{STFC}. This deployment has been managed by
the University of Alberta on behalf of CERN@school (and so \acs{IRIS}).
As full collaboration members, CERN@school also has access to 
the Timepix detector network managed by the \acf{CTU}, Prague.
Data from this network, which was operational during Runs I and II of
\acs{LHC} operations, has been used as the basis of studies by
\acs{IRIS} and \acs{MoEDAL} \acs{CERN} summer school students which
forms the basis of the \acs{MoEDAL} internal note,
``{\em On the flora and fauna of highly ionising background radiation
at Interaction Point 8}''.
Further details may be found in Section~\ref{meeting:moedaltimepixnote}.
%
\item \bullettext{The Nuclear Track Detectors and Monopole Quest!}:
as with all research programmes, new developments and opportunities
led to a slight shift in direction for the Fellow's activities.
While CERN@school's expertise originally focussed on the Timepix detector,
it was suggested that the thousands of visitors to the \ac{RSSSE} 
could be engaged with \acs{MoEDAL}'s research by helping to analyse
scanned images from the \acp{NTD}.
With this in mind,
a Zooniverse Citizen Science project was designed and created
by \acs{IRIS} students using the
Panoptes project builder. This enabled volunteers (at the \ac{RSSSE}
or elsewhere)
to identify and measure etch pits and other features in scans
from \ac{NTD} plastic exposed to the \acs{LHC} radiation environment
and heavy ion test beams.
The results from this formed the basis of a pilot study
reported in a \acs{MoEDAL} internal note
(see Section~\ref{meeting:moedalntdnote}).
Furthermore, a guide to MoEDAL (CAS-PUB-MDL-000002)
and accompanying educator presentation template (CAS-PUB-MDL-000004)
were prepared to facilitate the use of Monopole Quest! in
a classroom context. Details of these can be found in
Sections~\ref{meeting:moedalcasguide} and~\ref{meeting:moedalschtemplate}
respectively.
%
\item \bullettext{MoEDAL software and the Grid}:
towards the end of 2016, the Fellow also established the
\acs{MoEDAL} \acf{VO} on the \acf{WLCG} which enabled both
\acs{MoEDAL} and CERN@school researchers to make use of
GridPP's computing resources.
Following this, and a successful run of simulations used in the
analysis of 13 TeV data from the
\acf{MMT} subdetector system,
the Fellow was appointed Software Coordinator for the
\acs{MoEDAL} Collaboration.
As part of the duties associated with this role,
the MoEDAL Grid User Guide (CAS-PUB-MDL-000006)
and \code{moedal-run-simulations} repository
were created; these are detailed in Sections~\ref{meeting:moedalgriduserguide}
and~\ref{sec:moedalrunsimulations} respectively.
MoEDAL also featured as a case study for
GridPP's New User Engagement Programme.
MoEDAL (and \acs{LHCb}) software is available to
\acs{IRIS} students via \acs{CVMFS} on a GridPP CernVM.
\end{itemize}

As a member of the \acs{MoEDAL} Collaboration,
the Fellow was also included on the \acs{MoEDAL} author list.
Associated publications are listed in Section~\ref{sec:publications}.
These include the first physics results from the \acs{MoEDAL}
experiment as reported in the \acf{JHEP}~\cite{MoEDAL2016a}
and \acf{PRL}~\cite{MoEDAL2017a}.
The reader may also be interested in
the CERN@school bibliography (CAS-PUB-MDL-000001),
which contains a fairly comprehensive set of references
relating to MoEDAL and many other searches for magnetic monopoles.
Further details may be found in Section~\ref{meeting:moedalbib}.

\clearpage

%=============================================================================
\subsection{Documents}
\label{sec:moedaldocuments}
%=============================================================================
Table~\ref{tab:moedaldocuments} lists the documents relating to
\acs{MoEDAL} research and activities supported by \acs{IRIS}.
A brief description of each is provided 
in the following subsections.

\input{moedal/documents/documentstable}

\input{moedal/documents/documents}

\clearpage

%=============================================================================
\subsection{Presentations}
\label{sec:moedalpres}
%=============================================================================
Table~\ref{tab:moedalmeetings} lists the presentations given as part
of the Fellow's work with the \acs{MoEDAL} Collaboration.
Each one is then briefly described in the following
subsections.
%
Please note that the \acs{MoEDAL}
\acf{SaAG} % Software and Analysis group meetings
and \acs{MoEDAL} Collaboration meetings are only available to
\acs{MoEDAL} Collaboration members with the appropriate
credentials.
%
If access is required, please contact the
\acl{IRIS}\footnote{%
\url{http://researchinschools.org}}.

\input{moedal/presentations/presentationstable}

\clearpage

\input{moedal/presentations/presentations}

\clearpage

%=============================================================================
\subsection{Events}
\label{sec:moedalevents}
%=============================================================================
Table~\ref{tab:moedalevents} lists MoEDAL-related events that
have taken place since CERN@school first became involved with the
experiment.
The events themselves are then briefly described in the following subsections.
Note that for MoEDAL, this includes school visits, student workshops,
and other outreach and engagement-focussed events.
%
It should also be noted that the first MoEDAL Collaboration meeting
(also held at CERN) was not officially a collaboration meeting,
being more of a monopole physics workshop.  Regardless, it occured
before CERN@school was involved with the experiment so is not included
in the list of MoEDAL/CERN@school events.

\input{moedal/events/eventstable}

\clearpage

\input{moedal/events/events}

\clearpage

%=============================================================================
\subsection{Code repositories}
\label{sec:moedalcode}
%=============================================================================
The shared code repositories used for work on the \acs{MoEDAL} experiment
are listed in Table~\ref{tab:moedalcode}.
The code required to generate various public guides and documents 
has been archived on the CERN@school GitHub repository:

\href{http://github.com/CERNatschool}{http://github.com/CERNatschool}

Active analysis code and MoEDAL data cannot reside in the public domain
(at least until the associated results have been published by the
\acs{MoEDAL} collaboration). The corresponding repositories are therefore
hosted on \acs{CERN}'s GitLab system available to those with \acs{CERN}
computing accounts. \acs{MoEDAL}'s repositories are hosted here:

\href{https://gitlab.cern.ch/moedal/}{https://gitlab.cern.ch/moedal/}

Students and teachers who have got as far as requiring access to the
non-public MoEDAL data should contact \acs{IRIS} about obtaining a
\acs{CERN} computing account and joining the \acs{MoEDAL}
\acl{SaAG}.

\input{moedal/code/codetable}

\input{moedal/code/code}

\clearpage
